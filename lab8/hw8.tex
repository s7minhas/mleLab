\documentclass[10pt,letter]{article} 

%%%%%%%%%%%%%%%%%%%%%%%%%%%%%%%%%%%%%%%%%%%%%%%%%%
%%%%%%%%%%%%%%%%%%%% PREAMBLE %%%%%%%%%%%%%%%%%%%%
%%%%%%%%%%%%%%%%%%%%%%%%%%%%%%%%%%%%%%%%%%%%%%%%%%


% -------------------- defaults -------------------- %
% load lots o' packages

% layout control
\usepackage[paper=a4paper,left=25mm,right=25mm,top=30mm,bottom=25mm]{geometry}
\usepackage[onehalfspacing]{setspace}
\setlength{\parskip}{.5em}
\usepackage{rotating}
\usepackage{setspace}
\usepackage{fancyhdr}	
\usepackage{parallel}
\usepackage{parcolumns}

% math typesetting
\usepackage{array}
\usepackage{amsmath}
\usepackage{amssymb}
\usepackage{amsfonts}
\usepackage{relsize}

\usepackage{verbatim}

% restricts float objects to be inserted before end of section
% creates float barriers
\usepackage[section]{placeins}

% tables
\usepackage{tabularx}
\usepackage{booktabs}
\usepackage{multicol}
\usepackage{multirow}
\usepackage{longtable}

\usepackage[%
decimalsymbol=.,
digitsep=fullstop
]{siunitx}

% to adapt caption style
\usepackage[font={small},labelfont=bf]{caption}

% references
\usepackage[longnamesfirst]{natbib}
\bibpunct{(}{)}{;}{a}{}{,}

% footnotes at bottom
\usepackage[bottom]{footmisc}

% to change enumeration symbols begin{enumerate}[(a)]
\usepackage{enumerate}

% to make enumerations and itemizations within paragraphs or
% lines. f.i. begin{inparaenum} for (a) is (b) and (c)
\usepackage{paralist}

% to colorize links in document. See color specification below
\usepackage[pdftex,hyperref,x11names]{xcolor}

% for multiple references and insertion of the word "figure" or "table"
% \usepackage{cleveref}

% load the hyper-references package and set document info
\usepackage[pdftex]{hyperref}

% graphics stuff
\usepackage{subfig}
\usepackage{graphicx}
\usepackage[space]{grffile} % allows us to specify directories that have spaces
\usepackage{placeins} % prevents floats from moving past a \FloatBarrier
%\usepackage{tikz}
% \usepackage{pgfplots}

% define clickable links and their colors
\hypersetup{%
unicode=false,          % non-Latin characters in Acrobat's bookmarks
pdftoolbar=true,        % show Acrobat's toolbar?
pdfmenubar=true,        % show Acrobat's menu?
pdffitwindow=false,     % window fit to page when opened
pdfstartview={FitH},    % fits the width of the page to the window
pdfnewwindow=true,%
pagebackref=false,%
pdfauthor={Shahryar Minhas},%
pdftitle={Title},%
colorlinks,%
citecolor=black,%
filecolor=black,%
linkcolor=black,%
urlcolor=RoyalBlue4}

% Including External Code
\usepackage{verbatim}
\usepackage{listings}
\lstset{
	language=R,
	basicstyle=\scriptsize\ttfamily,
	commentstyle=\ttfamily\color{gray},
	numbers=left,
	numberstyle=\ttfamily\color{gray}\footnotesize,
	stepnumber=1,
	numbersep=5pt,
	backgroundcolor=\color{white},
	showspaces=false,
	showstringspaces=false,
	showtabs=false,
	frame=single,
	tabsize=2,
	captionpos=b,
	breaklines=true,
	breakatwhitespace=false,
	title=\lstname,
	escapeinside={},
	keywordstyle={},
	morekeywords={}
	}

% -------------------------------------------------- %


% -------------------- title -------------------- %

\title{}
\vspace{\baselineskip}
\author{}
\date{\today}

\setlength{\headheight}{15pt}
\setlength{\headsep}{20pt}
\pagestyle{fancyplain}
 
\fancyhf{}
 
\lhead{\fancyplain{}{POLSCI 733: Homework 8}}
\chead{\fancyplain{}{}}
\rhead{\fancyplain{}{\today}}
\rfoot{\fancyplain{}{\thepage}}

% ----------------------------------------------- %


% -------------------- customizations -------------------- %

% define the includegraphics search path
\graphicspath{{Graphics/}}

% easy commands for number propers
\newcommand{\first}{$1^{\text{st}}$}
\newcommand{\second}{$2^{\text{nd}}$}
\newcommand{\third}{$3^{\text{rd}}$}
\newcommand{\nth}[1]{${#1}^{\text{th}}$}

% easy command for boldface math symbols
\newcommand{\mbs}[1]{\boldsymbol{#1}}

% define bibliography style
\bibliographystyle{/Users/janus829/Documents/APSR}

% -------------------------------------------------------- %


%%%%%%%%%%%%%%%%%%%%%%%%%%%%%%%%%%%%%%%%%%%%%%%%%%
%%%%%%%%%%%%%%%%%%%% DOCUMENT %%%%%%%%%%%%%%%%%%%%
%%%%%%%%%%%%%%%%%%%%%%%%%%%%%%%%%%%%%%%%%%%%%%%%%%

% \doublespacing 

\begin{document}

This homework is due before class, March 24$^{th}$ 2015. 

\begin{itemize}
\item Do not use Zelig for this homework.
\item Problem 1: Create a generic function to estimate a logistic model using maximum likelihood.
	\begin{itemize}
		\item Use the examples in the lecture or textbook to create this function. For now assume that you have no missing data. 
		\item The inputs to this function should be a matrix of covariates and a vector of values for the dependent variable.
		\item The outputs of this function should contain a matrix of coefficients similar in structure to what was required in the midterm, a variance covariance matrix, and a check of whether or not the model converged.
	\end{itemize}
\item Problem 2: Use this function to analyze the Muller and Seligson data on insurgency and inequality in the following way:
	\begin{itemize}
		\item Download the file \textbf{Msrepl87.asc} from the dataverse at http://bit.ly/wccyDu as an ascii file.
		\item Load this file into your workspace as follows:
		\begin{verbatim}
		ms<-read.table("/Users/mw160/Downloads/Msrepl87.asc", header=TRUE, 
			     colClasses=c("character",rep("numeric" ,22)))
		rownames(ms) <- ms$country
		\end{verbatim}
		\item Make the following transformations:
		\begin{itemize}
			\item Create a variable, sanctions, that is the number of sanctions in 1970, per capita
			\item Create a variable, deaths, that is 1 if the number of deaths in 1975 was 1 or more
		\end{itemize}
		\item Use the function you wrote for Problem 1 to estimate a model that predicts binary deaths as a function of an intercept and the variable sanctions.
		\item Present the results in a coefficient plot. Write a paragraph summarizing the information you put in the table.
		\item Next undertake a 10-fold cross-validation of this model, and assess whether or not the coefficient estimate on the sanctions variable changes depending on the fold. Additionally, for each fold you hold out provide a visual depiction of the performance (e.g, through a ROC or separation plot or both) for the model you estimated leaving out that fold.
		\item Next determine the probability of there being a death in 1975 under six simple scenarios for the sanctions variable, specifically, the minimum, 1st quartile, median, mean, 3rd quartile, and maximum number of sanctions. 
		\item Last, extend this analysis by accounting for uncertainty in your model, using the strategies we developed in class and in the textbook. Provide a visual representation of this analysis.
		\item Write this all up in a 3-5 page memo, with tables and chairs included.
	\end{itemize}
\end{itemize}

\newpage

\end{document} 